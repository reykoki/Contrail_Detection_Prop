\section{Contrail Detection}
\label{sec:intro}

Contrail detection is an essential component towards understanding the impact and mitigation of contrails. Being able to detect contrails with high accuracy is a necessary capability in order to evaluate and compare contrail predictive models, to track contrails through their lifespan and measure the amount of heat that they trap, and ultimately to verify the effectiveness of flight path diversion efforts.

Remote sensing satellites, equipped with state-of-the-art sensors, such as the Advanced Baseline Imager (ABI) on the Geostationary Operational Environmental Satellites (GOES) \cite{goes} have allowed for reliable and consistent environmental observations. Compared to Polar-orbiting Operational Environmental Satellites (POES) such as the Suomi or Sentinel satellites, GOES maintains constant observation over a fixed area with relatively higher temporal resolution but at the cost of lower spatial resolution. With 2 kilometer spatial resolution in the infrared bands, GOES imagery in not sufficient to caputure the initial formation of young contrails, but is able to capture the more mature stages of contrails if they continue to spread out. GOES's coarser resolution narrows our focus on the contrails with the largest climate impact since persistent contrails that have expanded sufficiently to be observable at 2 kilometer resolution are associated with more significant warming effects \cite{warm}, \cite{persist}. Additionally, the proposed contrail predictive model will be at 3 kilometer resolution, thus reducing the utility of higher resolution contrail detection. 

In the last 20 years, contrails have primarily been detected in POES imagery using image processing techniques such as the Mannstein et al algorithm \cite{mannstein} that applies a series of hand engineered convolution and thresholding operations to infrared imagery. More recently, a method for tracking the lifecycle of contrails that uses the aforementioned Mannstein et al algorithm on POES imagery for early stage detection and then uses a tracking algorithm on GOES imagery for later stage contrail evolution \cite{track}. These methods all use POES detection 






